\documentclass{article}
\usepackage[utf8]{inputenc}
\usepackage{lmodern}
\usepackage{amsmath}
\usepackage{parskip}
% \usepackage{graphicx}
% \graphicspath{ {./} }
\title{Hal\_3900 Proposal}
\begin{document}
\begin{LARGE}
\begin{center}
\vspace*{15mm}

COMP3900

Hal\_3900 Proposal

\rule[4.5pt]{0.61\textwidth}{0.3pt}

% also need to include email and scrum role
\begin{align*}
  \text{Ellen Oates}    \quad   &\text{z5098896} \\
  \text{Hayden Le}      \quad   &\text{z5098972} \\
  \text{Yi Wang}        \quad   &\text{z5124282} \\
  \text{Zain Afzal}     \quad   &\text{z5059449} \\
\end{align*}

\rule[4.5pt]{0.61\textwidth}{0.3pt}

x/03/2019

\end{center}
\end{LARGE}
\newpage

%-------------------------------------------------------------------------------------------------%

\section{Background}
\subsection{The Problem}
* could we survey lecturers and tutors to get an idea of the quantity of emails they get / 
how long they spend answering administrative questions per week? would this help our proposal?

Online learning is changing the way students access and engage with higher education. Courses with 
online delivery increase the flexibility and accessibility of courses by providing students
with a platform to learn course content in their own time, at their own pace. Increasingly courses which are 
taught face to face include some online content delivery, including course materials, quizzes, online lecture recordings,
and forums to ask questions and discuss the course content outside of class.   

There is a delay between when students ask a question and when they get a response, and this can vary from hours to days.
Answering questions via email or on the forums takes a significant amount of time for tutors, course administrators and lecturers, 
and many questions could be effectively answered by an AI, much more quickly.

Some students need more learning support for course content (and don't always use the help sessions/tutorials effectively for this)
Some students need more support with lab work and assignment organisation to keep on top of due dates etc. 

* cite research about online learning from educator perspective
Lecturers and tutors want to know what their students need the most help with, to target their teaching better. Students are in different 
places with their learning and often need more individualised support to learn either the basic content, or suggestions of more advanced 
material to extend themselves. This is difficult to provide in a large class properly.

\subsection{The Existing Solution}
Currently at UNSW, learning support is provided to students through email, forums and help sessions. However many education
platforms that are predominantly online, are using a format that includes chat bot as a teaching assistant.

\subsection{Our Solution}
Our chat bot! * more to go ;)

\section{Epics}
\subsection{Answers administrative questions about course}


\subsection{Answers questions about course content}


\subsection{Meaningfully quizzes and interacts with students}


\subsection{Informs course admin/staff about cohort}


\subsection{Learns for more then one course without needing massive config}

\section{Epic Selection}

\section{Summary}


\end{document}
