\documentclass{article}
\usepackage[utf8]{inputenc}
\usepackage{lmodern}
\usepackage{amsmath}
\usepackage{parskip}
% \usepackage{graphicx}
% \graphicspath{ {./} }
\title{Hal\_3900 Proposal}
\begin{document}
\begin{LARGE}
\begin{center}
\vspace*{15mm}

COMP3900

Hal\_3900 Proposal

\rule[4.5pt]{0.61\textwidth}{0.3pt}

% also need to include email and scrum role
\begin{align*}
  \text{Ellen Oates}    \quad   &\text{z5098896} \\
  \text{Hayden Le}      \quad   &\text{z5098972} \\
  \text{Yi Wang}        \quad   &\text{z5124282} \\
  \text{Zain Afzal}     \quad   &\text{z5059449} \\
\end{align*}

\rule[4.5pt]{0.61\textwidth}{0.3pt}

x/03/2019

\end{center}
\end{LARGE}
\newpage

%-------------------------------------------------------------------------------------------------%

\section{Glossary}
\begin{align*}
  \text{Missed Question}    \quad   &\text{A question which the chat bot was not able to answer}
\end{align*}

\section{Background}
\subsection{The Problem}
* could we survey lecturers and tutors to get an idea of the quantity of emails they get / 
how long they spend answering administrative questions per week? would this help our proposal?

Online learning is changing the way students access and engage with higher education. Courses with 
online delivery increase the flexibility and accessibility of courses by providing students
with a platform to learn course content in their own time, at their own pace. Increasingly courses which are 
taught face to face include some online content delivery, including course materials, quizzes, online lecture recordings,
and forums to ask questions and discuss the course content outside of class.   

There is a delay between when students ask a question and when they get a response, and this can vary from hours to days.
Answering individual student questions via email or on the forums requires a significant amount of time for tutors, course administrators and lecturers. 
Often the same questions will be asked many times by different students, making it inefficient to have course staff respond to each one individually.

........ 

Some students need more learning support for course content (and don't always use the help sessions/tutorials effectively for this)
Some students need more support with lab work and assignment organisation to keep on top of due dates etc. 

* cite research about online learning from educator perspective
Lecturers and tutors want to know what their students need the most help with, to target their teaching better. Students are in different 
places with their learning and often need more individualised support to learn either the basic content, or suggestions of more advanced 
material to extend themselves. This is difficult to provide in a large class properly.

\subsection{The Existing Solution}
Currently at UNSW, learning support is provided to students through email, forums and help sessions. However many education
platforms that are predominantly online, are using a format that includes chat bot as a teaching assistant.
blah lots more .....

\subsection{Our Solution}
Our chat bot! * more to go ;)
Our goal is to create a chat bot companion for students to enhance their learning experience. The chat bot will provide students with learning support 
by responding to their questions about course administration and the content they are learning, in real time, and monitoring their understanding of the
course content with follow up questions. The chat bot will assist students with revision, and will provide organizational support by keeping students 
informed of their grades and upcoming due dates.

Our chat bot will enhance course delivery by keeping staff informed of their students' learning needs and frequently asked questions. It will reduce the 
load on lecturers, tutors and course administration staff by answering many of the questions that students have, allowing them to focus on the overall
delivery of the course. 


\section{Epics}
\subsection{Answers administrative questions about course}


\subsection{Answers questions about course content}


\subsection{Meaningfully quizzes and interacts with students}


\subsection{Informs course admin/staff about cohort}

This feature will allow the chat bot to maintain information about the questions users are asking and the performance of users in the excersizes the bot sets forward. This will be accessable from the admin web interface available to course staff.

The main metrics that should be captured for a requested time frame are
\begin{itemize}
  \item A list of the most common questions asked and their relative frequency
  \item A list of missed questions
  \item A list of the questions or triggers which the bot had a low satisifaction rate for
  \item A breakdown of every topic covered and it's retention rate amongst users (derived from interactive quizzing)
  \item General statistics on the bot's usage rate, uptime and the average computation time taken for a response
\end{itemize}

The web interface will provide the ability to 
\begin{itemize}
  \item View interactive sortable tablular data for the followed metrics
  \item View certain data such as usage rate and computation time as a graph against time
  \item Export the metrics as csv or json
  \item Register admin staff to be notified by email when certain conditions are met, i.e missed question rate rises past 50%
\end{itemize}

Out of scope will be any more advanced interaction with the data other then simple tabulation and graphing against time. More forms of data visualisation should be derferred to specialist tools. 

In addition this is simply to inform users of the bots performance and will not provide features to adjust the paramaters of the bot, this will be defered to the second epic involving the bot's interaction with course content. 

This will be a somewhat difficult set of features to implement, it's estimated difficulty score is 7/10 with a time estimate of 5 time units.

\subsection{Learns for more then one course without needing massive config}
The chat bot is able to adapt to other courses provided at UNSW and start providing student support quickly 
Course admin gets a really easy setup wizard?? think about.... 
(make it plug and play basically)   will improve the wording on this


\section{Epic Selection}

\section{Summary}


\end{document}
