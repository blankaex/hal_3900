\section{Introduction}

\subsection{The Problem}
Online learning is changing the way students access and engage with higher education. Courses with online delivery increase the flexibility and accessibility of education by providing students with a platform to learn course content in their own time, at their own pace. Increasingly courses which are taught face to face include some online content delivery, including course materials, quizzes, online lecture recordings, and forums to ask questions and discuss the course content outside of class. Because online learning is so prevalent in higher education, it is crucial for universities to ensure students are satisfied with their learning experience. 

There is a delay between when students ask a question and when they get a response, and this can vary from hours to days. Answering individual student questions via email or on forums requires a significant amount of time for tutors, course administrators and lecturers. Often the same questions will be asked many times by different students, making it inefficient to have course staff respond to each one individually.

Some key factors that contribute to student satisfaction in the online learning space are:
\begin{itemize}
  \item Students' preferences for actively participating in learning, rather than through passive learning styles.
  \item Students' expectations on instructors to facilitate their learning by organising the course resources\footnote{\url{https://www.researchgate.net/publication/282699144\_Student\_Satisfaction\_with\_Online\_Learning\_Is\_it\_a\_Psychological\_Contract}}
  \item The amount of interaction students have with each other, and the availability of their instructors.\footnote{'Key Factors for Determining Student Satisfaction in Online Courses': \url{https://www.learntechlib.org/primary/p/2226/article\_2226.pdf}}
  \item The availability of strong administrative support when using online learning tools or when confused about assessments and learning expectations.
  \item Course staff who are concerned with the quality of their course delivery, and want to know what their students need the most help with
  \item The availability of individual support and extended materials
\end{itemize}

Making these factors available to students becomes more challenging as classes grow in size. Course staff are thus in need of a more effective way to support with their students. 

\subsection{Existing Solutions \& Problems}
Currently at UNSW, learning support is provided to students through email, forums and help sessions. This is very man-hour intensive, requiring many tutors to be on hand to answer questions which, of themselves, are quite repetitive. In addition, as these courses become larger with increasing enrollment sizes, it becomes more difficult to be able to give students individual attention.

Another side effect of growing cohort sizes is the fact that many tutors and lecturers are forced to spend most of their time answering admin related questions, which is time that could be spent improving the course. The current solution has been to hire more staff and offload the majority of questions to forums, however these are full of repeated questions and require many tutors to moderate them.

This growth is becoming unsustainable, and with the rise of online education platforms, many students are eager to interact with course material in a more meaningful way. Waiting for a tutor to respond often creates a disconnect between the initial question and the answer, which limits the effectiveness of the response. This is provided that the tutor finds time to respond at all.

Chat bots have been deployed in some areas of secondary education, which interact with students in meaningful ways outside of class hours\footnote{\url{https://botsify.com/education-chatbot}}, and some have even been created to answer university-level questions\footnote{\url{https://www.canberra.edu.au/about-uc/media/newsroom/2018/february/students-make-new-friend-in-lucy-the-chatbot}}. However these tools can not be easily adopted by all university courses, or their administration and assessments.

\subsection{Our Solution}
\subsubsection{Our Contributions}
Our bot provides a platform for students to ask questions in real time. This addresses the issue of course staff being too busy to respond to questions in a timely fashion. This also grants lecturers and tutors more time to dedicate towards delivering and improving the course itself.

The bot goes beyond this by providing quiz functionality as well. Students can request to receive quizzes, and see questions and answers provided by the bot. This gives students meaningful interactions and a helpful study tool that doesn't draw from the time of course staff.

\subsubsection{Aim, Purpose \& Scope}
Our goal was to create a course companion chat bot for students to enhance their learning experience. The chat bot should provide students with support by responding to their questions about course administration and the content they are learning in real time. In addition, the bot will monitor students' understanding of the course content with follow-up questions. This is expected to increase both the amount and the quality of student interaction within the course. The bot will also provide students with more frequent and timely interactions. This helps by diverting more complex questions to tutors and lecturers, who in turn will have more time to respond to such questions in depth. 

The chat bot will provide administrative support by keeping students informed of their grades and upcoming due dates. This is expected to increase positively affect learning outcomes by boosting students' motivation to study. In addition, it will also enhance course delivery by keeping staff informed of their students' learning needs and frequently asked questions. It will reduce the load on course staff by answering many of the questions that students have and allowing them to focus on the overall delivery of the course. 

\subsubsection{Differences to Existing Systems}
The main difference between our bot and the existing solution of allocating the work to tutors is that the bot does not require an active human at all times. This addresses a number of problems addressed above, primarily saving time for course staff. The bot can be active throughout the day and respond to questions immediately, which is useful for both students and tutors, who previously had to email or post on forums at their convenience.

Our solution to this problem is innovative for a number of reasons. Most notably, we decided to create a bot that could complete what was previously considered human work. While machines replacing humans is no new idea, this is typically only the case in fields of physical labour. We have taken this a step further to create a bot that could supplant not the human muscles, but the human mind.

In addition, our bot also sets itself apart from other chat bots. Most common chatbots use an existing chat frontend such as Facebook Messenger or Slack. However, we have chosen to develop our own web interface from scratch. This gives us significantly more fine-grained control over the features of our bot, allowing us to tune the bot for our specific use case. This is expanded on in section 2.2.4.

\newpage
